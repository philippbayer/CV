%%%%%%%%%%%%%%%%%%%%%%%%%%%%%%%%%%%%%%%%%
% Friggeri Resume/CV
% XeLaTeX Template
% Version 1.0 (5/5/13)
%
% This template has been downloaded from:
% http://www.LaTeXTemplates.com
%
% Original author:
% Adrien Friggeri (adrien@friggeri.net)
% https://github.com/afriggeri/CV
%
% License:
% CC BY-NC-SA 3.0 (http://creativecommons.org/licenses/by-nc-sa/3.0/)
%
% Important notes:
% This template needs to be compiled with XeLaTeX and the bibliography, if used,
% needs to be compiled with biber rather than bibtex.
%
%%%%%%%%%%%%%%%%%%%%%%%%%%%%%%%%%%%%%%%%%

\documentclass[]{friggeri-cv} % Add 'print' as an option into the square bracket to remove colors from this template for printing

\addbibresource{bibliography.bib} % Specify the bibliography file to include publications

\begin{document}

\header{dr. philipp}{bayer}{postdoctoral researcher} % Your name and current job title/field

%----------------------------------------------------------------------------------------
%	SIDEBAR SECTION
%----------------------------------------------------------------------------------------

\begin{aside} % In the aside, each new line forces a line break
\section{contact}
\href{mailto:philippbay@gmail.com}{philippbay@gmail.com}
\href{http://github.com/philippbayer}{github.com/philippbayer}
\href{http://twitter.com/philippbayer}{twitter.com/philippbayer}
\section{languages}
German: mother tongue
English: fluent
French \& Japanese: advanced
\section{programming}
Python
Go, Perl, Bash, Java
Ruby on Rails, HTML
\section{research}
Genomics of complex traits in canola and wheat
\section{stats}
Citations: 492
h-index: 7
i10-index: 6
\end{aside}

%----------------------------------------------------------------------------------------
%	EDUCATION SECTION
%----------------------------------------------------------------------------------------

\section{education}

\begin{entrylist}
%------------------------------------------------
\entry
{2012--2015}
{PhD {\normalfont Applied Bioinformatics}}
{University of Queensland, Brisbane}
{Working in the Applied Bioinformatics group on the use of genotyping by sequencing to improve the genome assembly of canola. Thesis submission data: 23rd September 2015. Date of acceptance: 4th May 2016}
%------------------------------------------------
\entry
{2010--2012}
{Master {\normalfont of IT}}
{Bond University, Gold Coast}
{Graduated with Honours}
%------------------------------------------------
\entry
{2006--2009}
{Bachelor of Science {\normalfont Biology}}
{University of Münster, Germany}
{Thesis: Analysis of splicing in two populations of marine plants
using bioinformatic approaches}
\end{entrylist}

%----------------------------------------------------------------------------------------
%	WORK EXPERIENCE SECTION
%----------------------------------------------------------------------------------------

\section{experience}

\begin{entrylist}

\entry
{2017--Current}
{Hacky Hour Founder}
{UWA, Perth}
{Founded the Hacky Hour at UWA, a weekly get-together of researchers and staff working with programming and data, doubles as a help-desk for students with programming problems.}
%------------------------------------------------
\entry
{2017--Current}
{Mozilla Open Science Leadership mentor}
{UWA, Perth}
{Mentoring open source programmers and researchers on how to streamline and grow open source and open science projects under the umbrella of Mozilla.}
%------------------------------------------------
\entry
{2016--Current}
{EMBL-ABR Head of Nodes member, Open Science Special Interest Group member}
{UWA, Perth}
{EMBL-ABR is an Australian-wide network supporting the technical needs of life sciences researchers. Members of the group of Head of Nodes meet monthly to discuss the way forward for the organisation. The Open Science Special Interest Group meets bimonthly to discuss how EMBL-ABR can advance open science in Australia.}
%------------------------------------------------
\entry
{2016--Current}
{Postdoctoral researcher}
{UWA, Perth}
{Researching the genetics of complex plants with a focus on canola and wheat. Working closely with industry partners to improve their breeding programs. Preparing, writing, and publishing research. Currently supervising two interns, co-supervising four PhD students and one MSc student. Supervising the local computational infrastructure and data management. Assisting other researchers. Started and continue to run the group's journal club.}
%------------------------------------------------
\entry
{2013--Current}
{Software Carpentry and Data Carpentry instructor}
{Australia}
{Certified Software Carpentry and Data Carpentry instructor. Software Carpentry is a Mozilla/Alfred P. Sloan Foundation funded non-profit organization which teaches best programming practices (structured programming, reproducible research, version control etc.) in bootcamps to scientists around the world. Data Carpentry is a sister-organisation that focuses on teaching best data management practices.}
%------------------------------------------------
\end{entrylist}
% The entrylist is 'forced' to be one page, looks ugly
\begin{entrylist}
\entry
{2012--2012}
{Research exchange {\normalfont at Bayer CropScience}}
{Ghent, Belgium}
{For 4 weeks, worked on the assembly of the \textit{Brassica napus} genome. Learned to work in a corporate science environment. Still regularly involved with Bayer CropScience and continue to collaborate with the company on many projects.}
%------------------------------------------------
\entry
{2011--Current}
{Co-founder openSNP.org}
{Germany/Australia}
{A project for costumers of genotyping companies like 23andMe to share their data with scientists around the world, for free. Partially wrote and still maintain the site's Ruby on Rails code-base, interact and manage with the community of 5000 users, administration of the site's servers, and supervision of contributors.}

%------------------------------------------------

%------------------------------------------------
\end{entrylist}

%----------------------------------------------------------------------------------------
%	AWARDS SECTION
%----------------------------------------------------------------------------------------

\section{awards}

\begin{entrylist}
%------------------------------------------------
\entry
{2014}
{GRDC Travel Award}
{GRDC}
{\$4000 travel cost scholarship}
%------------------------------------------------
\entry
{2014}
{SAFS Travel Award}
{University of Queensland}
{\$2500 travel cost scholarship}
%------------------------------------------------
\entry
{2011--2014}
{Two postgraduate scholarships}
{University of Queensland}
{For the work on genotyping by sequencing, covers tuition and living costs.}
%------------------------------------------------
\entry
{2012}
{First place in PLOS/Mendeley Binary Challenge}
{Won with openSNP.org}
{Won first price in a competition aimed towards the advancement of open science}
%------------------------------------------------
\entry
{2009-2011}
{Master IT}
{Bond University}
{5x Top of class,  3x Vice-Chancellor's List for Academic Excellence, 1x IT Award Academic Excellence. Graduated with honours. Recipient of John Oglethorpe Medal for highest GPA of all IT students graduating that semester}
\end{entrylist}


%----------------------------------------------------------------------------------------
%	TEACHING SECTION
%----------------------------------------------------------------------------------------

\section{teaching}

\begin{entrylist}
%------------------------------------------------
\entry
{2017}
{Teaching Software Carpentry}
{Curtin University, Perth}
{Introduction to version control}
%------------------------------------------------
\entry
{2017}
{Teaching Software Carpentry}
{Research Bazaar, Curtin University, Perth}
{Introduction to data manipulation using Python}
%------------------------------------------------
\entry
{2016}
{Teaching and hosting Data Carpentry}
{UWA, Perth}
{Hosted, planned, and set up the first Data Carpentry workshop at UWA, taught best data management practices}
%------------------------------------------------
\entry
{2016, 2017}
{University teaching}
{UWA, Perth}
{Co-teach and co-supervise SCIE4002, computational analysis for biology and biomedical MSc students. Set up and maintain the computational infrastructure needed for practicals.}
%------------------------------------------------
\entry
{2016}
{Teaching Software Carpentry}
{Research Bazaar, Murdoch University, Perth}
{Taught introduction to Python}
%------------------------------------------------
\entry
{2016}
{Software Carpentry}
{UQ, Brisbane}
{Hosted, planned, and set up the first Software Carpentry workshop at UQ. Taught introduction to programming.}
%------------------------------------------------
\entry
{2013--2014}
{Software Carpentry}
{Adelaide/Melbourne}
{Assisted Software Carpentry bootcamp in Adelaide, taught basic to intermediate Python as well as documentation and assisted at bootcamp in Melbourne.}
%------------------------------------------------
\entry
{2009--2011}
{Tutoring}
{Bond University}
{Tutored students in Intro to Programming (Java), Database Management (Oracle/MySQL) and Networks \& Applications, held several all-day refresher courses before exams}
%------------------------------------------------
\end{entrylist}


\section{public speaking}
\begin{entrylist}
%------------------------------------------------
\entry
{2017}
{Presentation}
{COMBINE event, Perth}
{Title: Towards better plant breeding at UWA}
%------------------------------------------------
\entry
{2017}
{Presentation}
{Plant And Animal Genome conference, San Diego}
{Title: Improving Plant Breeding using KNetMiner}
%------------------------------------------------
\entry
{2016}
{Presentation}
{CCDM, Curtin University}
{Title: Towards a canola pan-genome: cautionary tales from the assembly bench}
%------------------------------------------------
\entry
{2015}
{Presentation}
{Plant And Animal Genome conference, San Diego}
{Title: Assessing and validating the amphidiploid genome of \textit{Brassica napus} using genotyping by sequencing}
%------------------------------------------------
\entry
{2014}
{Presentation}
{University of Queensland, GenGen Seminar Series}
{Title: Assembling and validating the genome of the \textit{Brassica napus} using skim-based genotyping by sequencing}
%------------------------------------------------
\entry
{2013}
{Poster}
{Plant And Animal Genome conference, San Diego}
{Title: Genome Assembly Validation and Trait Association using Skim Based Genotyping by Sequencing in Canola}
%------------------------------------------------
\entry
{2012}
{Presentation}
{28th Chaos Communication Congress, Berlin}
{Presented the work on openSNP, talked about the future of personal genomics and privacy implications}
%------------------------------------------------
\end{entrylist}

%----------------------------------------------------------------------------------------
%	PUBLICATIONS SECTION
%----------------------------------------------------------------------------------------

\section{publications}

\cite{bayer2017assembly}
- Conceived of the study, performed the majority of analysis steps, wrote the publication.\\

\cite{yuan2017improvements}
- Added several citations, added paragraphs, proof-read and error-corrected.\\

\cite{montenegro2017pangenome}
- Streamlined the annotation pipeline.\\

\cite{kaur2017advanced}
- Created the advanced ordering of assembly, performed new gene prediction, analysed cultivar differences.\\

\cite{gacek2016genome}
- Ran the genome-wide association study, interpreted results.\\

\cite{golicz2016pangenome}
- Ran different orders of the \textit{B. oleracea} assembly steps, ran annotation, helped with biological analysis and plotting.\\

\cite{hane2017comprehensive}
- Helped with setup of annotation and repeat pipeline.\\

\cite{barash2016candidate}
- Ran the entire genome-wide assocation study, interpreted results.\\

\cite{lee2016genome}
- Supervised part of the analysis, helped with biological interpretation.\\

\cite{bayer2016genomics}
- Collected literature, wrote the review.\\

\cite{visendi2016efficient}
- Helped with streamlining and error removal of the assembly pipeline, helped interpret results, plotting.\\

\cite{mason2016centromere}
- Analysed different alleles ratios between individuals, wrote analysis script to find outlier regions.\\

\cite{bayer2016skim}
- Collected literature, wrote the review.\\

\cite{bayer2015high}
- Conceived the study, carried out the majority of the research, wrote the majority of software, wrote publication.\\
 
\cite{golicz2015skim}
- Wrote parts of the review, collected literature.\\

\cite{lai2015identification}
- Helped with analysis and phylogenetic analysis.\\

\cite{chalhoub2014early}
- Carried out genotyping by sequencing analysis, manually validated contig placement, compiled report of misassemblies.\\

\cite{mason2014high}
- Performed the technical analysis and wrote the R-script.\\

\cite{greshake2014opensnp}
- Built large parts of the website, co-wrote the publication.\\

\cite{dattolo2013acclimation}
- Ran EST-based differential expression analysis of seagrasses grown in different depths.\\

%printbibsection{article}{articles in peer-reviewed journals} % Print all articles from the bibliography

%\printbibsection{book}{books} % Print all books from the bibliography

%\begin{refsection} % This is a custom heading for those references marked as "inproceedings" but not containing "keyword=france"
%\nocite{*}
%\printbibliography[sorting=chronological, type=inproceedings, title={international peer-reviewed conferences/proceedings}, notkeyword={france}, heading=subbibliography]
%\end{refsection}

%\begin{refsection} % This is a custom heading for those references marked as "inproceedings" and containing "keyword=france"
%\nocite{*}
%\printbibliography[sorting=chronological, type=inproceedings, title={local peer-reviewed conferences/proceedings}, keyword={france}, heading=subbibliography]
%\end{refsection}

%----------------------------------------------------------------------------------------

\end{document}
